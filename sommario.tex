\chapter*{Sommario} % senza numerazione
\label{sommario}

\addcontentsline{toc}{chapter}{Summary} % da aggiungere comunque all'indice

\label{cha:intro}
TODO: put in abstract?
Battery electric vehicles are becoming a fundamental part of the transportation industry. This technology shift brings new challenges in many fields of study. This thesis overviews some of these challenges regarding battery management systems on electric cars.\\
Batteries are complex electro-chemical components that need to be constantly monitored in order to ensure maximum safety and efficiency. Especially in the automotive domain, where vehicles are often subject to unoptimal and variable working conditions, a system that ensures the correct operation of the battery is necessary. Since the BMS can influence the performance of a vehicle, it is important to have a well designed and calibrated management system that isn't too invasive to the vehichle's operation, but retains acceptable levels of safety.

This thesis covers the challenges of developing a battery management system for a Formula SAE electric race car.\\ The document focuses on the error management software and cell balancing algorithms, explaining the implementation choices and analyzing experimental results.

Sommario è un breve riassunto del lavoro svolto dove si descrive l'obiettivo, l'oggetto della tesi, le
metodologie e le tecniche usate, i dati elaborati e la spiegazione delle conclusioni alle quali siete arrivati.

Il sommario dell’elaborato consiste al massimo di 3 pagine e deve contenere le seguenti informazioni:
\begin{itemize}
  \item contesto e motivazioni
  \item breve riassunto del problema affrontato
  \item tecniche utilizzate e/o sviluppate
  \item risultati raggiunti, sottolineando il contributo personale del laureando/a
\end{itemize}




